\documentclass{article}

\usepackage{polski}
\usepackage[utf8]{inputenc}
\usepackage{amsfonts}
\usepackage{amsthm}
\usepackage{amsmath}
\usepackage{babel}
\usepackage[style=dutch]{csquotes}

\title{Rozwiązanie zadania 78}
\author{Sławomir Górawski}
\date{\today}

\begin{document}

\maketitle

\noindent\textbf{Zadanie 78.}
Pokaż że dla każdego transducera Moore’a istnieje równoważny transducer Mealy’ego.
Pokaż że dla każdego transducera Mealy’ego istnieje \\
równoważny transducer Moore’a.

\paragraph{1.}
Weźmy dowolny transducer Moore'a $T$.
Pokażemy, że da się skonstruować dla niego transducer Mealy'ego $T'$
taki, że $f_{T'} = f_T$.

Idea rozwiązania opiera się na tym,
że transducer Moore'a generuje słowa przy wchodzeniu do nowych stanów,
a Mealy'ego -- podczas przechodzenia z jednego stanu do drugiego.
Jeśli mamy symbol $a$,
który powoduje przejście ze stanu $q_1$ do $q_2$,
to w transudcerze Moore'a wpływ na generowanie słowa ma $q_2$,
a w Mealy'ego -- $q_1$ i $a$.
Można dla każdemu takiemu przejściu w transducerze Mealy'ego $T'$
przypisać słowo generowane przez stan,
w którym znalazłby się transducer Moore'a $T$ po wykonaniu tego przejścia.

Formalnie: niech
$T = \langle \Sigma, \Sigma_1, Q, q_0, \delta, \sigma \rangle$.
Zdefiniujmy
$\sigma' : Q \times \Sigma \rightarrow \Sigma_1^*$
w następujący sposób:
$\sigma'(q, a) = \sigma(\delta(a, q))$.
Teraz możemy skonstruowac transducer Mealy'ego
$T' = \langle \Sigma, \Sigma_1, Q, q_0, \delta, \sigma' \rangle$.
Pokażemy indukcyjnie,
że $f_{T'}(w) = f_T(w)$ dla każdego $w \in \Sigma^*$ o długości $n$,
dla każdego $n \in \mathbb{N}$.

\begin{proof}
Podstawa indukcji: weźmy $n = 0$, jedyne słowo długości $0$ to $\varepsilon$,
$f_{T'}(\varepsilon) = \varepsilon = f_T(\varepsilon)$.
Krok indukcyjny: dla danego $n$ załóżmy,
że $f_{T'}(w) = f_T(w)$
dla każdego $w \in \Sigma^*$ długości $n$.
Weźmy dowolne $v \in \Sigma^*$ długości $n + 1$.
Następnie:

\begin{align*}
    f_{T'}(v) & = f_{T'}(wa)
    && \text{($a \in \Sigma$)} \\
    & = (f_{T'}(w))\sigma'(\hat{\delta}(w, q_0), a) \\
    & = (f_T(w))\sigma'(\hat{\delta}(w, q_0), a)
    && \text{(z założenia indukcyjnego)} \\
    & = (f_T(w))\sigma(\delta(a, \hat{\delta}(w, q_0))
    && \text{(z definicji $\sigma'$)} \\
    & = (f_T(w))\sigma(\hat{\delta}(wa, q_0)) \\
    & = f_T(wa) \\
    & = f_T(v)
\end{align*}
Pokazaliśmy, że $f_{T'}(v) = f_T(v)$
dla każdego $v \in \Sigma^*$ długości $n$, $n \in \mathbb{N}$.
\end{proof}

\newpage

\paragraph{2.}
Weźmy dowolny transducer Mealy'ego $T'$.
Pokażemy, że da się skonstruować dla niego transducer Moore'a $T''$
taki, że $f_{T''} = f_{T'}$.

Nawiązując do intuicji z $a$, $q_1$ i $q_2$ z poprzedniego punktu,
potrzebujemy w $q_2$ \enquote{zakodować} informacje z $a$ i $q_1$.
Aby uzyskać poprawne działanie $T''$ w ogólnym przypadku,
musimy rozszerzyć zbiór stanów $T'$ o informacje dot.
symboli, jakie zostały wczytane przy przejściu do danego stanu,
a konkretnie -- każdy stan $T''$ musi reprezentować to,
jaki stan był wcześniej,
jaki stan byłby w danym momencie w $T'$
i jaki symbol wywołał przejście.

Formalnie: niech
$T' = \langle \Sigma, \Sigma_1, Q', q_0', \delta', \sigma' \rangle$.
Wtedy
$T'' = \langle \Sigma, \Sigma_1, Q'', q_0'', \delta'', \sigma'' \rangle$, gdzie:

\begin{align*}
    & Q'' = Q' \times Q' \times \Sigma \\
    & q_0'' = \langle \_, q_0', \_ \rangle \\
    & \delta''(a, \langle \_, q, \_ \rangle)
    = \langle q, \delta'(a, q), a \rangle \\
    & \sigma''(\langle q, \_, a \rangle) = \sigma'(q, a)
\end{align*}
Konwencja: \enquote{\_} oznacza dowolną wartość odpowiedniego typu,
która nie jest istotna w danym kontekście.

Pokażemy indukcyjnie, że $T''$ i $T'$ są równoważne.

\begin{proof}
Podstawa indukcji -- jak w punkcie 1.
Krok indukcyjny: dla danego $n$ załóżmy,
że $f_{T'}(w) = f_T(w)$
dla każdego $w \in \Sigma^*$ długości $n$.
Weźmy dowolne $v \in \Sigma^*$ długości $n + 1$.
Następnie:

\begin{align*}
    f_{T''}(v) & = f_{T''}(wa)
    && \text{($a \in \Sigma$)} \\
    & = (f_{T''}(w))\sigma''(\hat{\delta''}(wa, q_0)) \\
    & = (f_{T'}(w))\sigma''(\hat{\delta''}(wa, q_0))
    && \text{(z założenia indukcyjnego)} \\
    & = (f_{T'}(w))\sigma''(\delta''(a, \hat{\delta''}(w, q_0)) \\
    & = (f_{T'}(w))\sigma''(\delta''(
        a, \langle \_, \hat{\delta'}(w, q_0), \_ \rangle
    )) \\
    & = (f_{T'}(w))\sigma''(
        \langle \hat{\delta'}(w, q_0), \_, a \rangle
    )
    && \text{(z definicji $\delta''$)} \\
    & = (f_{T'}(w))\sigma'(\hat{\delta}(w, q_0), a)
    && \text{(z definicji $\sigma''$)} \\
    & = f_{T'}(wa) \\
    & = f_{T'}(v)
\end{align*}
Pokazaliśmy, że $f_{T''}(v) = f_{T'}(v)$
dla każdego $v \in \Sigma^*$ długości $n$, $n \in \mathbb{N}$.
\end{proof}

\end{document}
