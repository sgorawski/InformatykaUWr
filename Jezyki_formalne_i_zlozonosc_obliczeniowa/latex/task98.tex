\documentclass{article}

\usepackage[utf8]{inputenc}
\usepackage{polski}
\usepackage{amsmath}
\usepackage{amssymb}
\usepackage{amsthm}
\usepackage[boxed]{algorithm}
\usepackage{algorithmic}
\usepackage{nopageno}
\usepackage{hyperref}

\newcommand{\N}{\mathbb{N}}
\newcommand{\reduces}{\le_\mathit{rek}}
\renewcommand{\complement}{\overline}

\begin{document}

\begin{flushright}
\emph{Sławomir Górawski}
\end{flushright}

\noindent\textbf{Zadanie 98.}
\emph{(Hierarchia arytmetyczna)}.
Niech $f \colon \N \times \N \to \N$
będzie pewną ustaloną obliczalną bijekcją.
Oznaczmy klasę zbiorów rekurencyjnych jako $\Sigma_0$.
Dla danego $\Sigma_i$
niech $\Pi_i = \{ A \subseteq \N \mid \N \setminus A \in \Sigma_i \}$,
zaś $A \in \Sigma_{i + 1}$,
jeśli istnieje $B \in \Pi_i$
takie że $A = \{ n \in \N \mid \exists m f(n, m) \in B\}$.
Niech $L$ będzie zbiorem numerów tych niepustych funkcji rekurencyjnych,
których dziedzina jest skończona.
Jakie jest najmniejsze $i$ dla którego zachodzi $L \in \Sigma_i$?

\paragraph{Rozwiązanie.}
Odpowiedź to 2. Pokażemy, że:
\begin{enumerate}
    \item $L \in \Sigma_2$,
    \item $L \notin \Sigma_1$.
\end{enumerate}
Intuicja z:
\url{https://www.wikiwand.com/en/Tarski-Kuratowski_algorithm}.

\paragraph{1.}
Niech $B$ będzie zbiorem takich $f(n, f(m, k))$,
że $n$ jest numerem funkcji rekurencyjnej,
której największym elementem dziedziny jest $m$
i $\phi_n(m)$ zwraca wynik w $k$ krokach.

Pokażemy, że $B \in \Pi_1$.
$\Pi_1$ to rodzina zbiorów co-r.e.
Pokażemy, że $B \reduces \complement{K}$.
Niech $r \colon \N \to \N$ będzie redukcją,
która dla $x = f(n, f(m, k))$ (znamy je, bo $f$ jest bijekcją)
zwraca numer takiego programu:

\begin{algorithm}[H]
    \begin{algorithmic}[1]
        \STATE wczytaj \_
        \STATE uruchom $k$ kroków $\phi_n(m)$
        \STATE jeżeli nie otrzymano wyniku, zwróć 1
        \STATE inteligentnie uruchom $\phi_n(i)$ dla każdego $i > m$, \\
            jeżeli coś zwrócił to zwróć 1
    \end{algorithmic}
\end{algorithm}

\begin{proof}
    Pokażemy, że $x \in B \iff r(x) \in \complement{K}$.
    
    \begin{enumerate}
        \item Dla $x = f(n, f(m, k)) \in B$:
            \begin{itemize}
                \item $\phi_n(m)$ zwróci wynik,
                \item $\phi_n(i)$ nic nie zwróci dla każdego $i > m$,
                \item program się zapętli,
                \item $r(x) \in \complement{K}$.
            \end{itemize}
        \item Dla $x = f(n, f(m, k)) \notin B$ mamy 2 możliwości:
            \begin{enumerate}
                \item $\phi_n(m)$ nie zwróci wyniku i program zwróci 1,
                \item $\phi_n$ zwróci wynik dla $m$
                    oraz dla jakiegoś $i > m$ i program zwróci 1.
            \end{enumerate}
            
            W obu przypadkach $r(x) \notin \complement{K}$.
    \end{enumerate}
\end{proof}

\paragraph{2.}
$\Sigma_1$ to rodzina zbiorów r.e.
Pokażemy, że $\complement{K} \reduces L$.
Niech $r \colon \N \to \N$ będzie redukcją,
która dla $n$ zwraca numer takiego programu:

\begin{algorithm}[H]
    \begin{algorithmic}[1]
        \STATE wczytaj m
        \STATE jeżeli $m = 1$, zwróć 1
        \STATE uruchom $\phi_n(n)$
        \STATE zwróć 1
    \end{algorithmic}
\end{algorithm}

\begin{proof}
    Pokażemy, że $n \in \complement{K} \iff r(n) \in L$.
    
    \begin{enumerate}
        \item Dla $n \in \complement{K}$:
            \begin{itemize}
                \item $\phi_n(n)$ nie zatrzyma się nigdy,
                \item $\phi_{r(n)}$ dla 1 zwróci 1
                    a dla innych argumentów się zapętli,
                \item dziedzina $\phi_{r(n)}$ to $\{ 1 \}$,
                \item $r(n) \in L$.
            \end{itemize}
        \item Dla $n \notin \complement{K}$:
            \begin{itemize}
                \item $\phi_n(n)$ zwróci wynik,
                \item dla każdego argumentu $\phi_{r(n)}$ zwróci 1,
                \item dziedzina $\phi_{r(n)}$ to $\N$,
                \item $r(n) \notin L$.
            \end{itemize}
    \end{enumerate}
\end{proof}

\end{document}
