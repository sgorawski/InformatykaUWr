\chapter{Description and analysis of the subject}

To decide on the best approach to the challenge,
first an analysis of desired properties and features of a solution
shall be conducted.
After that the technologies that the project will have to work with
will be looked into, with the goal to determine how are they similar,
and at the same time, how they differ.

\section{Definition of the goal}

To compile a list of features that ought to be demanded
from a crypto wallet management application,
we can compare it to services used by more traditional
internet banking.
As other applications of the blockchain technology,
for example smart contracts,
are out of this thesis' scope,
focus is to be on financial management.

The core functionality that the project cannot be considered complete without is to be:
\begin{itemize}
    \item creating cryptocurrency wallets,
    \item checking the amount of funds,
    \item making transfers.
\end{itemize}
All of the above should function in a similar way
from the end user's perspective
across all supported cryprocurrencies.

With these implemented an user should have sufficient control
over their crypto assets,
allowing him to participate in either personal or proffesional
financial operations over the cryptocurrency networks.

Some additional features, that would be convenient to have,
however will not be considered essential could be:
\begin{itemize}
    \item saving wallets' balance history,
    \item insight into value of the crypto assets (e.g. in US dollars),
    \item ability to import wallets used previously with other services.
\end{itemize}

The project should provide all this functionality
while remaining stable and portable,
prove resistant to predictable user errors
and ensure that managed assets' security will not be compromised.
The user experience ought to be simple,
and the interface should be clean and focused on the most important elements,
with function taking precedence over form.

\section{Technology overview}

The project will have to integrate with two blockchain generations:
first generation, represented by Bitcoin, the oldest and most known
cryptocurrency, and second generation, represented by Ethereum.
Although these cryptocurrencies have a lot in common
-- they both are based on blockchain, after all
-- they differ even in what they were to be.

Bitcoin was created as a peer-to-peer electronic cash system,
aiming to provide an alternative to traditional banking
-- performed through financial institutions
-- with a concept of a global,
distributed ledger of transactions,
secured from being mangled with by proof-of-work
in form of using computational power to mine blocks of transactions,
thus preventing them from being retroactively changed.

The premise of Ethereum was a bit different,
as it was focused more on distributed computation,
aiming to deliver what can be thought of as a decentralized computer,
with global state made of accounts containing information such as balance,
being secured by the blockchain technology.
The solution used currently for mining blocks is also proof-of-work,
although that is said to change in near future.

\subsection{Similarities}

Both technologies are interacted with through asymmetric cryptography
-- users are identified by their addresses and authorize operations
by signing data with their private keys.
In the core essence, a cryptocurrency wallet is all about its private key.
A public key is coupled with the private key,
and then an address is derived from the public key
as an additional layer of security.
This is true for both blockchains
-- even though the format of keys and addresses differs,
underlying principle remains the same.

\subsection{Differences}

As was hinted before, Ethereum accounts are stateful by nature,
Bitcoin ones, however, are not.
This difference is apparent when certain information needs to be
recovered from the blockchain.
It is often the case when it's saved as part of the Ethereum network's
global state and the extraction is straightforward.
Bitcoin is not based on a notion of any global state,
so in theory crawling an entire blochchain
might be required for reconstructing the current situation.

A good example of this is wallet balance
-- a part of the account state in Ethereum, easy to recover.
Bitcoin does not however deal with accounts directly,
it deals only with transactions;
so this quite basic piece of information might require
tracing all transactions somehow associated with a given wallet,
possibly many years into the history of Bitcoin.

\subsection{Conclusions}

Seamless integration of the project with both blockchains,
with details abstracted away under a shared interface
might prove a nontrivial task,
as there are quite substantial
differences in how these function.

However, the similarities between Bitcoin and Ethereum
make the same task seem sensible and worth the effort,
as for an end user dealing with cryptocurrencies,
both are in their essence just digital assets,
so providing an unified way to manage them
sounds like an useful idea.
