\chapter{For programmers}

The primary programming language used in the project is Python.
It was chosen because of its conciseness,
a vast array of open source third-party tools available,
such as libraries and frameworks,
and support for multiple programming paradigms,
such as imperative and object-oriented,
with functional and declarative elements.

The source code was kept in the Git \cite{git} version control system
and written using PyCharm \cite{pycharm} tool by JetBrains
on an educational license,
which can be too heavy for a small project,
but provides excellent support for code navigation and refactoring
in larger ones.

\section{Library}

\subsection{Structure}

The library consists of two major classes,
named \texttt{Bitcoin} and \texttt{Ethereum}.
They provide a unified interface for performing operations
on respective blockchains by exposing the following methods:
\begin{itemize}
    \item \texttt{create\_wallet},
    \item \texttt{get\_balance},
    \item \texttt{send\_money},
\end{itemize}
which should be self-descriptive.

There is also a class that provides an abstraction over the networking details,
which are pretty much the same in both blockchains,
as was described before.
The class is called \texttt{RpcProxy} and upon being instantiated with an URL
to an appropriate network node,
is passed to \texttt{Bitcoin} or \texttt{Ethereum} by dependency injection \cite{di}.
This allows for significant code reuse between respective currencies' classes.

\subsection{Tools} \label{6:lib:tools}

The following third-party dependencies were used to provide the library's functionality:
\begin{itemize}
    \item \texttt{pywallet} -- a library used for creating both Bitcoin and Ethereum wallets,
    chosen because of its universality,
    \item \texttt{eth-account} -- a collection of Ehtereum-specific utilities,
    used mostly for signing transactions with private keys
    (in Bitcoin this was done by the node itself),
    \item \texttt{requests} -- a tool used for making HTTP requests,
    offering a minimalistic interface better than its equivalents from the standard library;
    one of the most widely used Python libraries and a de facto standard in the industry.
\end{itemize}

For development, a unit testing framework \texttt{pytest} was also used,
providing a clean, declarative style interface superior to the standard library.
This one is not required for the library to work, though.

To test the integration with actual cryptocurrency networks,
the nodes were run as Docker containers using the Docker Compose utility,
exposing their ports for the manual and unit tests to use.
For Bitcoin, the image used was an official implementation Bitcoin Core \cite{bitcoin-core}
available as a Docker image \texttt{ruimarinho/bitcoin-core}.
Ethereum was tested against Ganache \cite{ganache},
a tool designed specifically for such use cases,
available as a Docker image \texttt{trufflesuite/ganache-cli}.

The entire library is completely independent of the web application;
it is packaged with \texttt{setuptools} to be used as if it were an external dependency.

\section{Web Application}

\subsection{Structure}

The application consists of:

\begin{itemize}
    \item request handling functions,
    \item a module exposing the library classes dynamically by currency,
    \item a module for integration with the Coinlayer API,
    \item an ORM mapping database relations to Python objects,
    \item templates for response rendering,
    \item database migrations stored as Python modules.
\end{itemize}

\subsection{Tools}

The application was created using Flask \cite{flask}, a popular Python web framework.
It is a modular tool which does not enforce any opninionated project structure,
which was desirable due to this project not being a typical CRUD backend.
Said popularity was another major advantage,
as there is now a vast array of guides and answered questions on the web.

The ORM and toolkit used for database operations is SQLAlchemy \cite{sa},
also one of the most popular Python tools of its category.
It allows for easy query construction and integrates seamlessly with Flask
thanks to the \texttt{flask-sqlalchemy} plugin.
Another plugin was \texttt{flask-migrate},
used to generate and apply database migrations with Alembic \cite{alembic}.

Web forms were created with the help of WTForms \cite{wtforms}
and its respective plugin \texttt{flask-wtf}.
This tool makes generating HTML easier, and it comes with built-in
input validation.

The \texttt{requests} library described
in \ref{6:lib:tools} was used to fetch cryptocurrency prices
from the Coinlayer API.

PostgreSQL \cite{postgres} was chosen as a database for the project,
due to popularity and its useful features.
For development, it was set up as a Docker container
along with testnet nodes and ran with Docker Compose.

HTML responses were generated with Jinja2 templating engine,
which is bundled with Flask as a sensible default option.
For styling and responsiveness Bootstrap \cite{bootstrap} CSS \cite{css} framework was used,
chosen because of popularity, ease of use and good looking design.

Last but not least, charts for viewing balances and values over time
were generated in JavaScript,
with the help of the Chart.js \cite{chartjs} library.
