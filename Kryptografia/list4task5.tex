\documentclass{article}

\usepackage[utf8]{inputenc}
\usepackage{polski}
\usepackage{amsmath}
\usepackage{amssymb}
\usepackage{nopageno}
%\usepackage{fullpage}

\begin{document}

\begin{flushright}
\textit{Sławomir Górawski}
\end{flushright}
\bigskip

\noindent\textbf{Lista 4, zadanie 5.}
Pokaż jak można rozłożyć na dwa czynniki liczbę złożoną $n$,
która w teście Millera--Rabina okazała się złożona,
ponieważ dla pewnego $a$ wyliczyliśmy
$a^{2^{k}r} \not\equiv \pm 1, a^{2^{k + 1}r} \equiv 1$ modulo $n$.

\paragraph{Rozwiązanie.}
Niech $x = a^{2^{k}r}$; wtedy $x^2 = a^{2^{k + 1}r}$. Wiemy, że:
\begin{align*}
    x \not\equiv \pm 1 &\mod n, \\
    x^2 \equiv 1 &\mod n.
\end{align*}
Z pierwszej równości wiemy, że $n$ nie dzieli $x + 1$ ani $x - 1$.
Przekształcając drugą, otrzymujemy:
\[
    (x + 1)(x - 1) \equiv 0 \mod n.
\]
Jeśli $n$ nie dzieli żadnego z powyższych,
to oba muszą zawierać w swoim rozkładzie czynniki $n$
niebędące 1 ani $n$. Możemy je znaleźć,
wyliczając $\gcd(n, x + 1)$ oraz $\gcd(n, x - 1)$.

\end{document}

