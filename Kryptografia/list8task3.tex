\documentclass{article}

\usepackage[utf8]{inputenc}
\usepackage{polski}
\usepackage{amsmath}

\usepackage{nopageno}

\begin{document}

\begin{flushright}
\textit{Sławomir Górawski}
\end{flushright}

\paragraph{Lista 8, zadanie 3.}
Mamy $n$ osób. Każda data urodzenia jest równie prawdopodobna
(załóżmy, że nikt nie urodził się 29 lutego).
Ile musi wynosić $n$,
żeby prawdopodobieństwo istnienia dwóch osób z tą samą datą urodzenia
przekraczało 1/2?

\paragraph{Rozwiązanie.}
Niech $P(n)$ oznacza prawdopodobieństwo,
że w grupie $n$ osób istnieją dwie osoby z tą samą datą urodzenia.
Policzymy najpierw prawdopodobieństwo zdarzenia przeciwnego:
każda z $n$ osób urodziła się innego dnia.
Aby tak było, pierwsza osoba może urodzić się dowolnego dnia,
druga -- każdego oprócz tego co pierwsza,
trzecia -- każdego oprócz tych co pierwsza i druga itd.:
\[
    P(n) = 1 - 1 \cdot
        \left( 1 - \frac{1}{365} \right) \cdot
        \left( 1 - \frac{2}{365} \right)
        \cdots
        \left( 1 - \frac{n - 1}{365} \right)
\]
Podstawiając kolejne wartości $n$,
możemy dojść do rozwiązania: $P(n) > 1/2$ dla $n \ge 23$.
Możemy też spróbować uzyskać wynik,
przybliżając wzór na $P(n)$ przy użyciu szeregu Taylora.
Skorzystamy z faktu, że $e^x \approx 1 + x$.
Wtedy $1 - a/365 \approx e^{-a/365}$.
Po podstawieniu daje nam to:
\begin{align*}
    P(n) &\approx 1 - e^{-1/365} \cdots e^{-(n - 1)/365} \\
        &= 1 - e^{-(1 + 2 + \cdots + (n - 1))/365} \\
        &= 1 - e^{-(n(n - 1)/2)/365} \\
        &= 1 - e^{-n(n - 1)/730} \\
        &\approx 1 - e^{-n^2/730}
\end{align*}
Chcemy teraz rozwiązać nierówność:
\begin{align*}
    P(n) &> \frac{1}{2} \\
    1 - e^{-n^2/730} &> \frac{1}{2} \\
    e^{-n^2/730} &< \frac{1}{2} \\
    -\frac{n^2}{730} &< -\ln 2 \\
    n^2 &> 730 \ln 2 \\
    n &> \sqrt{730 \ln 2} \\
    n &> 22.49\dots
\end{align*}
Dla całkowitych $n$ daje to $n \ge 23$.
\end{document}

