\documentclass{article}

\usepackage[utf8]{inputenc}
\usepackage{polski}
\usepackage{amsmath}
\usepackage{amssymb}
\usepackage{amsthm}
\usepackage{nopageno}
\usepackage{fullpage}

\newcommand{\xor}{\oplus}

\begin{document}

\begin{flushright}
\textit{Sławomir Górawski}
\end{flushright}
\bigskip

\noindent\textbf{Lista 2, zadanie 5.}
Przy przesyłaniu szyfrogramu w DES nastąpiło przekłamanie jednego bitu.
Ile bitów tekstu jawnego zostało utraconych jeśli DESa użyto w trybie
ECB, CBC, CFB, OFB, $k$-CFB, $k$-OFB, CTR.

\paragraph{Rozwiązanie.}
Niech $m_i$ będzie $i$-tym blokiem tekstu jawnego, a $C_i$ -- szyfrogramu.
Załóżmy, że to właśnie tam nastąpiło przekłamanie.
Wtedy, w zależności od trybu:

\begin{itemize}
    \item ECB: w tym trybie $m_i = D_K(C_i)$,
        zatem utraciliśmy tekst jawny tylko w $i$-tym bloku,
        ponieważ żadne inne nie zależą od $C_i$.
        Utracony został co najmniej jeden bit,
        bo nie jest możliwe, żeby $D_K$ dało taki sam wynik dla dwóch różnych wartości.
        (Tutaj: $C_i$ przekłamanym bitem i bez).
        Możemy oszacować liczbę utraconych bitów na 1--64.
    \item CBC: tutaj $m_i = D_K(C_i) \xor C_{i - 1}$.
        Tracimy 1--64 bitów tekstu jawnego w $i$-tym bloku, jak wyżej.
        $C_i$ ma również wpływ na wartość $m_{i + 1}$,
        jednak jest tylko XORowane z odszyfrowanym $C_{i + 1}$,
        więc w bloku $i + 1$ tekstu jawnego tracimy 1 bit.
    \item CFB: $m_i = E_K(C_{i - 1}) \xor C_i$.
        Rozumując jak wyżej, tracimy 1 bitów w $m_i$ i 1--64 bitów w $m_{i + 1}$.
    \item OFB i CTR: to szyfry strumieniowe.
        Niech $r_i$ -- $i$-ty klucz wygenerowany przez generator pseudolosowy,
        wtedy $C_i = m_i \xor r_i$.
        Zakładając, że klucz do generatora pseudolosowego jest nienaruszony,
        tracimy 1 bit tekstu jawnego, ponieważ $m_i = C_i \xor r_i$, a $r_i$ jest poprawne.
    \item $k$-CFB: $m_i$ to $C_i$ zXORowane
        z $k$ bitami zaszyfrowanego bloku $C_{i - 1}, C_{i - 2}, \dots$.
        W $i$-tym bloku tracimy 1 bit, natomiast $C_i$ ma wpływ jeszcze na kilka kolejnych ($64/k$).
        Nie wiemy, ile błędów pojawi się w $k$ ostatnich bitach zaszyfrowanego bloku
        podczas kolejnych rund deszyfrowania -- maksymalnie $k$ --
        więc możemy oszacować z góry liczbę utraconych bitów
        w blokach od $i + 1$ w górę jako $k \cdot 64/k = 64$.
    \item $k$-OFB: jak w OFB, $m_i = C_i \xor r_i$,
        zakładamy poprawność $r_i$, więc tracimy 1 bit tekstu jawnego.
\end{itemize}

\end{document}
