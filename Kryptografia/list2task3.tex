\documentclass{article}

\usepackage[utf8]{inputenc}
\usepackage{polski}
\usepackage{amsmath}
\usepackage{amssymb}
\usepackage{amsthm}
\usepackage{nopageno}
\usepackage{fullpage}

\newcommand{\xor}{\oplus}
% c for complement, shorter
\renewcommand{\c}{\overline}

\begin{document}

\begin{flushright}
\textit{Sławomir Górawski}
\end{flushright}
\bigskip

\noindent\textbf{Lista 2, zadanie 3.}
Niech $\c{x}$ będzie logicznym dopełnieniem ciągu x złożonego z zer i jedynek.
Niech $E$ oznacza szyfrowanie DESem.
Pokaż, że jeśli $y = E_K(x)$, to $\c y = E_{\c K}(\c x)$.
Jak używając tej tożsamości można zredukować dwukrotnie liczbę szyfrowań
przy kryptoanalizie DESa poprzez przeszukanie przestrzeni
kluczy dla danej pary tekst jawny -- szyfrogram?

\paragraph{1.}
Weźmy dowolny 64-bitowy ciąg $x$.
Pokażemy, że dla danego klucza $K$ jeśli $y = E_K(x)$, to $\c y = E_{\c K}(\c x)$.
W tym celu przyjrzyjmy się, co dzieje się z tekstem jawnym przy szyfrowaniu DESem:
\begin{enumerate}
    \item Na początku ciąg jest permutowany i dzielony na dwie 32-bitowe części: $l_0$ i $r_0$.
    \item Następnie obie części przechodzą przez 16 cykli szyfrowania --
        niech $(l_i, r_i)$ będą wynikiem $i$-tego cyklu.
    \item Po wykonaniu tych operacji $l_{16}$ oraz $r_{16}$ łączone są w jeden ciąg
        i dokonywana jest permutacja końcowa, która daje szyfrogram.
\end{enumerate}

Niech $L_i$ i $R_i$ będą wynikami $i$-tego cyklu szyfrowania tekstu $x$ kluczem $K$.
Permutacje zachowują dopełnienie,
tj. dla dowolnej permutacji bitów $\pi$ zachodzi $\pi(\c x) = \c{\pi(x)}$.
Wynika z tego, że aby wynikiem permutacji końcowej był ciąg $\c y$,
musi ona dostać na wejściu $\c{L_{16}}$ i $\c{R_{16}}$.
Pokażemy, że przy szyfrowaniu tekstu $\c x$ kluczem $\c K$
wynikiem $i$-tego cyklu szyfrowania są $\c{L_i}$ oraz $\c{R_i}$,
dla każdego $i$ od 0 do 16.

\begin{proof}
    Na początku pokażemy, że $l_0 = \c{L_0}$ i $r_0 = \c{R_0}$.
    Wiemy, że permutacje bitów zachowują dopełnienie,
    więc jeśli wejściem do permutacji początkowej był $\c x$,
    to jej wynikiem będzie $(\c{L_0}, \c{R_0})$.
    Teraz chcemy pokazać, że każdy cykl szyfrowania zachowuje dopełnienie.
    Załóżmy, że $l_i = \c{L_i}$ oraz $r_i = \c{R_i}$.
    Pokażemy, że wynikami kolejnego cyklu szyfrowania są wtedy $\c{L_{i + 1}}$ oraz $\c{R_{i + 1}}$.
    W DESie otrzymuje się je tak:
    \begin{align*}
        l_{i + 1} &= r_i, \\
        r_{i + 1} &= l_i \xor f(k_i \xor w(r_i)),
    \end{align*}
    gdzie $k_i$ oraz $w$ wyjaśnimy niżej. Dla lewej strony dowód jest krótki:
    \[
        l_{i + 1} = r_i = \c{R_i} = \c{L_{i + 1}}.
    \]
    
    Klucz $k_i$ jest tworzony przez przestawienie bitów i wybór 48 z nich.
    Ta operacja zachowuje dopełnienie;
    jeśli przy szyfrowaniu kluczem $K$ mieliśmy $k_i = k$, to obecnie $k_i = \c k$.
    Do rozszerzenia prawej części wejścia używamy permutacji rozszerzonej $w$,
    polega to na permutacji ze zduplikowaniem niektórych bitów;
    to również zachowuje dopełnienie. Razem daje to:
    \begin{align*}
        r_{i + 1} &= l_i \xor f(k_i \xor w(r_i)) \\
            &= \c{L_i} \xor f(\c k \xor w(\c{R_i})) \\
            &= \c{L_i} \xor f(\c k \xor \c{w(R_i)}) \\
            &= \c{L_i} \xor f(k \xor w(R_i)) && (\c a \xor \c b = a \xor b) \\
            &= \c{L_i \xor f(k \xor w(R_i))} && (\c a \xor b = \c{a \xor b}) \\
            &= \c{R_{i + 1}}.
    \end{align*}
    Nie musimy tutaj wiedzieć dokładnie, jak działa $f$ --
    wystarczy nam, że wynik jest taki sam,
    jak przy szyfrowaniu tekstu $x$ kluczem $K$.
    Z zasady indukcji $l_i = \c{L_i}$ i $r_i = \c{R_i}$ dla $i \in \mathbb{N}$.
    Nie wiemy, czym jest $k_i$ dla $i > 16$,
    jednak istotne dla nas jest,
    że $l_{16} = \c{L_{16}}$ oraz $r_{16} = \c{R_{16}}$.
\end{proof}

\paragraph{2.}
Aby wykorzystać udowodnioną tożsamość,
by zredukować liczbę szyfrowań przy kryptoanalizie DESa,
potrzebujemy dwóch par tekst jawny -- szyfrogram:
$(x, c_1)$ oraz $(\c x, c_2)$.
Następnie sprawdzamy po kolei potencjalne klucze --
dla danego $K$ obliczamy $y = E_K(x)$.
Następnie:
\begin{enumerate}
    \item Jeśli $y = c_1$, to $K$ jest szukanym kluczem.
    \item Jeśli $y = \c{c_2}$, to szukanym kluczem jest $\c{K}$.
    \item Jeśli żadna z równości nie jest prawdziwa,
        możemy odrzucić $K$ i $\c K$ z przestrzeni poszukiwań.
\end{enumerate}
W ten sposób jesteśmy w stanie przy pomocy jednego szyfrowania
sprawdzić dwa potencjalne klucze naraz.
Ciągów bitów mogących być kluczami jest $2^{56}$,
ale wystarczy $2^{55}$ szyfrowań by sprawdzić je wszystkie.

\end{document}
