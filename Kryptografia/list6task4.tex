\documentclass{article}

\usepackage[utf8]{inputenc}
\usepackage{polski}
\usepackage{amsmath}
\usepackage{amssymb}
\usepackage{nopageno}
%\usepackage{fullpage}

\begin{document}

\begin{flushright}
\textit{Sławomir Górawski}
\end{flushright}

\paragraph{Lista 6, zadanie 6.}
Alicja chce przesłać tę samą wiadomość $m$
do Boba, Charliego i Davida za pomocą kryptosystemu RSA.
Załóżmy, że $e_B = e_C = e_D = 3$ dla różnych $n_B, n_C, n_D$.
Pokaż, jak Oskar może odszyfrować
wiadomość $m$ po przechwyceniu jej szyfrogramów.

\paragraph{Rozwiązanie.}
Niech $c_B, c_C, c_D$ będą szyfrogramami, a $x = m^3$. Wtedy:
\begin{align*}
    x \equiv c_B &\mod n_B \\
    x \equiv c_C &\mod n_C \\
    x \equiv c_D &\mod n_D
\end{align*}
Załóżmy, że $n_B, n_C, n_D$ są względnie pierwsze.
Wtedy możemy znaleźć wartość $x$, która to spełnia,
przy użyciu chińskiego twierdzenia o resztach.
Z twierdzenia tego wynika,
że wszystkie rozwiązania przystają do siebie modulo $N = n_Bn_Cn_D$.
Weźmy najmniejsze możliwe $x$
(powinniśmy byli znaleźć właśnie takie, a jeśli nie, to weźmy $x$ modulo $N$).
W RSA $m$ musi być mniejsze niż $n_B$, $n_C$ i $n_D$.
Mamy $x \equiv m^3 \mod N$, gdzie $x < N$ oraz $m^3 < N$.
W takim razie $m = \sqrt[3]{x}$.

\end{document}
