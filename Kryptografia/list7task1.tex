\documentclass{article}

\usepackage[utf8]{inputenc}
\usepackage{polski}
\usepackage{amsmath}
\usepackage{amssymb}
\usepackage{nopageno}

\begin{document}

\begin{flushright}
\textit{Sławomir Górawski}
\end{flushright}

\paragraph{Lista 7, zadanie 1.}
(Założenie DL) Niech $p = 2q + 1$ ($p, q$–pierwsze) i $g$ rzędu $q$ w $\mathbb{Z}_p^*$.
Załóżmy, że istnieje wielomianowy algorytm probabilistyczny
obliczający dla losowych $g, g^x$ liczbę $x$
z prawdopodobieństwem co najmniej $1/w(l)$
gdzie $w$ jest wielomianem a $l$ długością $p$.
Skonstruuj wielomianowy algorytm probabilistyczny
obliczający $x$ dla zadanych $g, g^x$
z prawdopodobieństwem 0.9999.

\paragraph{Rozwiązanie.}
Niech $A$ będzie istniejącym algorytmem z zadania.
Konstruujemy nowy algorytm $B$ z parametrem $m$:
\begin{verbatim}
    dla i od 1 do m
        uruchom A
        jeśli otrzymano wynik, zwróć go
\end{verbatim}
Musimy dobrać $m$ tak żeby prawdopodobieństwo otrzymania wyniku przez $B$
było większe lub równe 0.9999.
Patrząc na zdarzenia przeciwne,
$B$ nie zwróci wyniku jeśli $A$ nie zwróci wyniku w żadnej z $m$ prób.
Prawdopodobieństwo tego to $(1 - \frac{1}{w(l)})^m$ i chcemy, żeby było mniejsze lub równe 0.00001.
Możemy z tego wyliczyć $m$:
\begin{align*}
    \left( 1 - \frac{1}{w(l)} \right)^m &\le 10^{-5} \\
    \ln \left( \left( 1 - \frac{1}{w(l)} \right)^m \right) &\le \ln(10^{-5}) \\
    m \ln \left( 1 - \frac{1}{w(l)} \right) &\le -5 \ln 10
\end{align*}
Korzystamy teraz z nierówności $1 - \frac{1}{x} \le \ln x$ dla $x > 0$:
\begin{align*}
    m \left( 1 - \frac{1}{1 - \frac{1}{w(l)}} \right) &\le -5 \ln 10 \\
    m \left( \frac{-1}{w(l) - 1} \right) &\le -5 \ln 10 \\
    -m &\le -5 \ln(10)(w(l) - 1) \\
    m &\ge 5 \ln(10)(w(l) - 1)
\end{align*}
Widzimy że $5 \ln 10$ jest stałą, więc $m \ge O(w(l))$.
Algorytm $B$ uruchamia $m$ razy wielomianowy algorytm $A$.
Jego złożoność to iloczyn wielomianów, więc sam też jest wielomianowy.
\end{document}
