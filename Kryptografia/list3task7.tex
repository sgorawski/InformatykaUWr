\documentclass{article}

\usepackage[utf8]{inputenc}
\usepackage{polski}
\usepackage{amsmath}
\usepackage{amssymb}
\usepackage{amsthm}
\usepackage{nopageno}
\usepackage{mathtools}
\usepackage[top=1in,bottom=1in,left=1.875in,right=1.875in]{geometry}

\newcommand{\xor}{\oplus}

\begin{document}

\begin{flushright}
\textit{Sławomir Górawski}
\end{flushright}
\bigskip

\noindent\textbf{Lista 3, zadanie 7.}
W AES S-boks oblicza wartość bajtu $b$ w ten sposób, że
\begin{itemize}
    \item Jeśli $b \ne 0$, to $c = b^{-1}$ w ciele $F_{2^8}$,
        które jest ciałem wielomianów nad $\mathbb{Z}_2$ z działaniami modulo
        nierozkładalny wielomian ósmego stopnia. Jeśli $b = 0$, to $c = 0$.
    \item Niech $c = c_0c_1c_2c_3c_4c_5c_6c_7$.
        Wtedy $d_i = c_i \xor c_{i + 4} \xor c_{i + 5} \xor c_{i + 6} \xor c_{i + 7}$,
        gdzie indeksy dodawane są modulo 8.
    \item Wynikiem działania S-boksa jest $e = d \xor 01100011$.
\end{itemize}
Jak wygląda przekształcenie odwrotne do tego S-boksa?

\paragraph{Rozwiązanie.}
Na początku odwrócimy dwa ostatnie kroki.
Można je przestawić następująco jako operacje na macierzach modulo 2:
\[
    e = Ac + t,
\]
gdzie $c = \begin{bsmallmatrix} c_0 & c_1 & \cdots & c_7 \end{bsmallmatrix}^T$,
$t = \begin{bsmallmatrix} 1 & 1 & 0 & 0 & 0 & 1 & 1 & 0 \end{bsmallmatrix}^T$ oraz
% Copied from Wikipedia source
\[
    A = \begin{bmatrix}
        1 & 0 & 0 & 0 & 1 & 1 & 1 & 1 \\
        1 & 1 & 0 & 0 & 0 & 1 & 1 & 1 \\
        1 & 1 & 1 & 0 & 0 & 0 & 1 & 1 \\
        1 & 1 & 1 & 1 & 0 & 0 & 0 & 1 \\
        1 & 1 & 1 & 1 & 1 & 0 & 0 & 0 \\
        0 & 1 & 1 & 1 & 1 & 1 & 0 & 0 \\
        0 & 0 & 1 & 1 & 1 & 1 & 1 & 0 \\
        0 & 0 & 0 & 1 & 1 & 1 & 1 & 1
    \end{bmatrix}.
\]
Odwracamy to przekształcenie:
\begin{align*}
    e &= Ac + t \\
    Ac &= e + t \\
    c &= A^{-1}(e + t) \\
    c &= A^{-1}e + A^{-1}t.
\end{align*}
Po odwróceniu macierzy $A$ modulo 2 otrzymujemy:
% Copied from Wikipedia source
\[
    A^{-1} = \begin{bmatrix}
        0 & 0 & 1 & 0 & 0 & 1 & 0 & 1 \\
        1 & 0 & 0 & 1 & 0 & 0 & 1 & 0 \\
        0 & 1 & 0 & 0 & 1 & 0 & 0 & 1 \\
        1 & 0 & 1 & 0 & 0 & 1 & 0 & 0 \\
        0 & 1 & 0 & 1 & 0 & 0 & 1 & 0 \\
        0 & 0 & 1 & 0 & 1 & 0 & 0 & 1 \\
        1 & 0 & 0 & 1 & 0 & 1 & 0 & 0 \\
        0 & 1 & 0 & 0 & 1 & 0 & 1 & 0
    \end{bmatrix},
\]
natomiast $A^{-1}t = \begin{bsmallmatrix} 1 & 0 & 1 & 0 & 0 & 0 & 0 & 0 \end{bsmallmatrix}^T$.
Możemy to zapisać jako:
\[
    c_i = e_{i + 2} \xor e_{i + 5} \xor e_{i + 7} \xor 00000101.
\]
Następnie musimy odwrócić pierwszy krok S-boksa,
co nic w tym przypadku nie zmienia --
szukamy odwrotności $c$ w $F_{2^8}$,
jeśli $c \ne 0$, to $b = c^{-1}$,
natomiast jeśli $c = 0$, to $b = 0$.
\end{document}
