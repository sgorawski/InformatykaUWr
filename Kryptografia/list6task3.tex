\documentclass{article}

\usepackage[utf8]{inputenc}
\usepackage{polski}
\usepackage{amsmath}
\usepackage{amssymb}
\usepackage{nopageno}

\begin{document}

\begin{flushright}
\textit{Sławomir Górawski}
\end{flushright}

\paragraph{Lista 6, zadanie 3.}
Alicja chce przesłać tę samą wiadomość $m$ do Boba i Charliego za pomocą kryptosystemu RSA.
Bob i Charlie używają tego samego $n$,
ale różnych wykładników klucza jawnego $e_B$ i $e_C$.
Załóżmy ponadto, że $\gcd(e_B, e_C) = 1$.
Pokaż, jak Oskar może odszyfrować wiadomość $m$
po przechwyceniu jej szyfrogramów przeznaczonych dla Boba i Charliego.
Czy daje to mu możliwość odtworzenia kluczy deszyfrujących?

\paragraph{Rozwiązanie.}
Niech $c_B$ i $c_C$ będą przechwyconymi szyfrogamami.
Wiemy, że:
\begin{align*}
    m^{e_B} \equiv c_B &\mod n, \\
    m^{e_C} \equiv c_C &\mod n.
\end{align*}
Używając rozszerzonego algorytmu Euklidesa,
możemy znaleźć takie $s_B$ i $s_C$ że:
\[
    e_Bs_B + e_Cs_C = \gcd(e_B, e_C) = 1.
\]
Wtedy możemy odszyfrować wiadomość $m$ w następujący sposób:
\[
    c_B^{s_B} \cdot c_C^{s_C} \equiv m^{e_Bs_B} \cdot m^{e_Cs_C} \equiv m \mod n.
\]
Jedna z liczb $s_B$, $s_C$ jest ujemna (załóżmy że $s_C$),
więc w praktyce chcemy policzyć:
\[
    (c_C^{-1})^{-s_C} \mod n,
\]
co wymusza dodatkowe założenie, że $\gcd(c_C, n) = 1$.
Czy daje to możliwość odtworzenia kluczy deszyfrujących?
Nie, aby mieć taką możliwość, Oskar musiałby otrzymać wiadomość $m$
zaszyfrowaną tym samym $n$ oraz wykładnikiem $e_O$,
dla którego znany jest mu klucz deszyfrujący $d_O$.
Wtedy mógłby rozłożyć $n$ i znaleźć $d_B$ oraz $d_C$.
Bez tego jednak nie dysponuje żadną informacją,
która mogłaby nam ułatwić odtworzenie kluczy deszyfrujących.

\end{document}
