\documentclass{article}

\usepackage[utf8]{inputenc}
\usepackage{polski}
\usepackage{amsmath}
\usepackage{amssymb}
\usepackage{nopageno}
%\usepackage{fullpage}

\begin{document}

\begin{flushright}
\textit{Sławomir Górawski}
\end{flushright}
\bigskip

\noindent\textbf{Lista 4, zadanie 4.}
W jaki sposób znając dwie reszty $b, c$ modulo $n$
takie $b \ne \pm c \mod n$ i $b^2 = c^2 \mod n$
można znaleźć rozkład $n$ na dwa czynniki.

\paragraph{Rozwiązanie.}
Przekształcając drugą równość, otrzymujemy:
\begin{align*}
    b^2 = c^2 &\mod n \\
    b^2 - c^2 = 0 &\mod n \\
    (b + c)(b - c) = 0 &\mod n.
\end{align*}
Wiemy z założenia, że ani $b + c$, ani $b - c$ nie przystają do 0 modulo $n$, ale ich iloczyn tak.
W takim razie w rozkładzie na czynniki oba muszą mieć w sobie dzielniki $n$.
Jeden z szukanych czynników to $\gcd(b - c, n)$.
Możemy go znaleźć np. algorytmem Euklidesa.
Drugi możemy znaleźć, dzieląc $n$ przez pierwszy.

\end{document}

